% -------------------------------------------------------------------------
% Setup
% -------------------------------------------------------------------------
\documentclass[11pt, aspectratio=149]{beamer}
% Options for aspectratio: 1610, 149, 54, 43 and 32, 169
\usepackage[utf8]{inputenc}
\usepackage[english]{babel}% Alternative: 'norsk'
\usepackage[expansion=false]{microtype}% Fixes to make typography better
\usecolortheme{beaver} % Decent options: beaver, rose, crane
\usepackage{listings}% To include source-code
\usepackage{booktabs}% Professional tables
\usefonttheme{serif}
\usepackage{mathptmx}
\usepackage[scaled=0.9]{helvet}
\usepackage{courier}

\title{Introduction to Kernel Density Estimation}
\subtitle{A graphical tutorial}
\date{\today}
\author{tommyod @ GitHub}

% -------------------------------------------------------------------------
% Package imports
% -------------------------------------------------------------------------
\usepackage{etoolbox}
\usepackage{graphicx}
\usepackage{tikz}
\usepackage{amsmath}
\usepackage{amsthm}
\usepackage{amsfonts}
\usepackage{amssymb}
\usepackage{mathtools}
\usepackage{graphicx}
\usepackage{hyperref}
\usepackage{listings}
\usepackage[sharp]{easylist}
\usepackage{multicol}
\usepackage{tikz-cd}

\usefonttheme{professionalfonts}
\usepackage{fontspec}
\setmainfont{Open Sans}
\setsansfont{Open Sans}
\setmonofont{Ubuntu Mono}
\usefonttheme{serif}

%gets rid of bottom navigation bars
\setbeamertemplate{footline}[frame number]{}

%gets rid of bottom navigation symbols
\setbeamertemplate{navigation symbols}{}

% Set up colors to be used
\definecolor{purered}{RGB}{31,119,180}
\definecolor{titlered}{RGB}{31,119,180}
\definecolor{bggray}{RGB}{242,242,242}
\definecolor{bggraydark}{RGB}{217,217,217}

% Change the default colors

\setbeamercolor*{title}{bg=bggray,fg=titlered}
\AtBeginEnvironment{theorem}{%
	\setbeamercolor{block title}{fg=titlered, bg=bggraydark}
	\setbeamercolor{block body}{fg=black,bg=bggray}
}
\AtBeginEnvironment{proof}{%
	\setbeamercolor{block title}{bg=bggraydark}
	\setbeamercolor{block body}{fg=black,bg=bggray}
}
\AtBeginEnvironment{example}{%
	\setbeamercolor{block title example}{bg=bggraydark}
	\setbeamercolor{block body example}{fg=black,bg=bggray}
}
\AtBeginEnvironment{definition}{%
	\setbeamercolor{block title}{bg=bggraydark}
	\setbeamercolor{block body}{fg=black,bg=bggray}
}

\setbeamercolor{block title example}{bg=bggraydark}
\setbeamercolor{block body example}{fg=black,bg=bggray}
\setbeamercolor{block title}{bg=bggraydark}
\setbeamercolor{block body}{fg=black,bg=bggray}

\setbeamercolor{frametitle}{fg=titlered,bg=bggray}
\setbeamercolor{section in head/foot}{bg=black}
\setbeamercolor{author in head/foot}{bg=black}
\setbeamercolor{date in head/foot}{fg=titlered}


% Spacing for lsits
\newcommand{\listSpace}{0.4em}

% Theorems, equations, definitions setup
\theoremstyle{plain}

\usepackage{etoolbox}
\usepackage{lipsum}

\makeatletter
\patchcmd{\beamer@sectionintoc}
{\vfill}
{\vskip\itemsep}
{}
{}
\makeatother  

\AtBeginSection[]{
	\begin{frame}
		\vfill
		\centering
		\begin{beamercolorbox}[sep=8pt,center,shadow=false,rounded=false]{title}
			\usebeamerfont{title}\insertsectionhead\par%
		\end{beamercolorbox}
		\vfill
	\end{frame}
}

% -------------------------------------------------------------------------
% Document start
% -------------------------------------------------------------------------
\begin{document}
\maketitle

\section{Long term strength progression}

\begin{frame}[fragile, t]{TITLE}
	\vfill
	TODO
	\vfill
	\begin{figure}
		\centering
		\includegraphics[width=\linewidth]{strength_progression_longterm.pdf}
	\end{figure}
	\vfill
\end{frame}

\begin{frame}[fragile, t]{TITLE}
	\vfill
	TODO
	\vfill
	\begin{figure}
		\centering
		\includegraphics[width=\linewidth]{strength_progression_program_specific.pdf}
	\end{figure}
	\vfill
\end{frame}


\begin{frame}[fragile, t]{TITLE}
	\vfill
	TODO
	\vfill
	\begin{figure}
		\centering
		\includegraphics[width=\linewidth]{strength_progression_program_sawtooth_varying_nonlinearity.pdf}
	\end{figure}
	\vfill
\end{frame}


\begin{frame}[fragile, t]{TITLE}
	\vfill
	TODO
	\vfill
	\begin{figure}
		\centering
		\includegraphics[width=\linewidth]{strength_progression_program_sawtooth_varying_period.pdf}
	\end{figure}
	\vfill
\end{frame}

\begin{frame}[fragile, t]{TITLE}
	\vfill
	TODO
	\vfill
	\begin{figure}
		\centering
		\includegraphics[width=\linewidth]{strength_progression_program_sawtooth_varying_scale.pdf}
	\end{figure}
	\vfill
\end{frame}


\begin{frame}[fragile, t]{TITLE}
	\vfill
	TODO
	\vfill
	\begin{figure}
		\centering
		\includegraphics[width=\linewidth]{strength_progression_program_sinusoidal_varying_period.pdf}
	\end{figure}
	\vfill
\end{frame}

\section{Volume and intensity}

\begin{frame}[fragile, t]{TITLE}
	\vfill
	TODO
	\vfill
	\begin{figure}
		\centering
		\includegraphics[width=\linewidth]{periodization_reps_intensity.pdf}
	\end{figure}
	\vfill
\end{frame}


\begin{frame}[fragile, t]{TITLE}
	\vfill
	TODO
	\vfill
	\begin{figure}
		\centering
		\includegraphics[width=\linewidth]{periodization_reps_intensity_double.pdf}
	\end{figure}
	\vfill
\end{frame}

\begin{frame}[fragile, t]{TITLE}
	\vfill
	TODO
	\vfill
	\begin{figure}
		\centering
		\includegraphics[width=\linewidth]{periodization_reps_intensity_out_of_phase.pdf}
	\end{figure}
	\vfill
\end{frame}


\section{Sets and repetitions}


\begin{frame}[fragile, t]{TITLE}
	\vfill
	TODO
	\vfill
	\begin{figure}
		\centering
		\includegraphics[width=\linewidth]{reps_intensity_mapping.pdf}
	\end{figure}
	\vfill
\end{frame}


\begin{frame}[fragile, t]{TITLE}
	\vfill
	TODO
	\vfill
	\begin{figure}
		\centering
		\includegraphics[width=\linewidth]{reps_intensity_mapping_many.pdf}
	\end{figure}
	\vfill
\end{frame}


\section{Repetition schemes}


\begin{frame}[fragile, t]{TITLE}
	\vfill
	If sets of $\{2, 3, 4, 5 \}$ reps are allowed, how can we create a scheme with $12$ reps in total?
	\vfill
There are 12 ways, and they are:
	\vfill
	\begin{align*}
	(2, 2, 2, 2, 2, 2) &&
	(4, 2, 2, 2, 2) \\
	(3, 3, 2, 2, 2) &&
	(5, 3, 2, 2) \\
	(4, 4, 2, 2) &&
	(4, 3, 3, 2) \\
	(5, 5, 2) &&
	(3, 3, 3, 3) \\
	(5, 4, 3) &&
	(4, 4, 4)
	\end{align*}
	\vfill
\end{frame}



\begin{frame}[fragile, t]{TITLE}
	\vfill
	If sets of $\{3, 4, 5, 6, 7, 8 \}$ reps are allowed, how can we create a scheme with $25$ reps in total?
	\vfill
	There are 45 ways, some of which are:
	\vfill
	\begin{align*}
(4, 3, 3, 3, 3, 3, 3, 3)  &&
(8, 5, 3, 3, 3, 3)  \\
\mathbf{(8, 8, 3, 3, 3)} &&
(7, 7, 5, 3, 3) \\
\mathbf{(7, 6, 5, 4, 3)} &&
(8, 8, 6, 3)  \\
(5, 4, 4, 4, 4, 4)  &&
(8, 8, 5, 4) \\
(5, 5, 5, 5, 5) &&
(7, 6, 6, 6) 
	\end{align*}
	\vfill
\end{frame}


\begin{frame}[fragile, t]{TITLE}
	\vfill
	If sets of $\{3, 4, 5, 6, 7, 8 \}$ reps are allowed, how can we create a scheme with $25$ reps in total?
	\vfill
	There are 45 ways, some of which are:
	\vfill
	\begin{align*}
	(4, 3, 3, 3, 3, 3, 3, 3) \, [92.7] &&
	(8, 5, 3, 3, 3, 3)\,  [87.0] \\
	\mathbf{(8, 8, 3, 3, 3)}\,  [83.4] &&
	(7, 7, 5, 3, 3) \, [85.0] \\
	\mathbf{(7, 6, 5, 4, 3)} \, [85.7] &&
	(8, 8, 6, 3)\, [81.2] \\
	(5, 4, 4, 4, 4, 4) \, [89.3] &&
	(8, 8, 5, 4) \, [81.6] \\
	(5, 5, 5, 5, 5)\,  [86.8] &&
	(7, 6, 6, 6)\,  [82.9]
	\end{align*}
	\vfill
\end{frame}







\end{document}
